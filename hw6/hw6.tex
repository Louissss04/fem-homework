\documentclass[12pt,a4paper]{article}
\usepackage{geometry}
\geometry{left=2.5cm,right=2.5cm,top=2.0cm,bottom=2.5cm}
\usepackage[english]{babel}
\usepackage{amsmath,amsthm}
\usepackage{amsfonts}
\usepackage[longend,ruled,linesnumbered]{algorithm2e}
\usepackage{fancyhdr}
\usepackage{ctex}
\usepackage{array}
\usepackage{listings}
\usepackage{color}
\usepackage{graphicx}
\usepackage{amssymb}
\newtheorem{theorem}{定理}
\newtheorem{lemma}[theorem]{引理}
\newtheorem{corollary}[theorem]{推论}

\begin{document}
	
	\noindent
	
	\section*{2024.05.13}	
	
	\begin{enumerate}
		\item 给出三次Lagrange元的局部插值误差 $\left\|u-\Pi_K u\right\|_{\ell, K}(0 \leq \ell \leq 1)$ 估计结果
		
		$$
		\|u - \Pi_K u\|_{0,K} \leq C h^4 |u|_{4,K} $$
		$$
		\|u - \Pi_K u\|_{1,K} \leq C h^3 |u|_{4,K}
		$$
		
		\item 给出三次Lagrange元求解Poisson问题的解的最优能量 $\left(H^1\right)$ 范数误差估计 $\left(\right.$ 最高几阶)、凸区域情形的 $L^2$ 范数误差估计结果, 具有最高正则性假设下 (即给出 $s$ 的最大值)的负范数估计.
		
		$H^1$误差估计:
		$$
		\left\|u-u_h\right\|_{1,\Omega}\leq Ch^3\left|u\right|_{4,\Omega}
		$$
		
		$L^2$误差估计:(假设区域光滑或凸)
		$$
		\left\|u-u_h\right\|_{0,\Omega}\leq Ch\left\|u-u_h\right\|_{1,\Omega}\leq Ch^4\left|u\right|_{4,\Omega}
		$$
		
		负范数估计:(假设正则性、逼近性)
		$$
		\left\|u-u_h\right\|_{-2,\Omega}\leq Ch^{3}\left\|u-u_h\right\|_{1,\Omega}\leq Ch^6\left|u\right|_{4,\Omega}
		$$
		
		\item 利用对偶论证的方法证明重调和问题中的 $H^1$ 误差估计定理
		
		\begin{proof}
			设 $\phi_g \in V$ 是变分问题
			$$
			a\left(v, \phi_g\right)=(g, v), \forall v \in V
			$$
			
			的解. 有
			$$
			\begin{aligned}
				\left(g, u-u_h\right) & =a\left(u-u_h, \phi_g\right) \\
				& = a\left(u-u_h, \phi_g-v_h\right) \qquad \qquad  \forall v_h \in V_h \\
				& \leq\left\|u-u_h\right\|_{2, \Omega} \inf _{v_h \in V_h}\left\|\phi_g-v_h\right\|_{2, \Omega}
			\end{aligned}
			$$
			
			再利用
			$$
			\left\|u-u_h\right\|_{1, \Omega}=\sup _{0 \neq g \in H^{-1}(\Omega)} \frac{\left(g, u-u_h\right)}{\|g\|_{-1, \Omega}}
			$$
		\end{proof}
		
		\item 证明二阶积分公式
		$$
		\int_K \phi \mathrm{dx} \approx \frac{S_K}{3} \sum_{i=4}^6 \phi\left(a_i\right)
		$$
		
		对 $\phi \in P_2(K)$ 精确成立
		
		\begin{proof}
			令
			$$
				\phi_i=\lambda_i(2\lambda_i-1),\quad1\leq i\leq3
			$$	
			$$
			\phi_4=4\lambda_2\lambda_3,\phi_5=4\lambda_3\lambda_1,\phi_6=4\lambda_1\lambda_2
			$$
			
			对任意多项式$\phi\in P_2(K)$,有
			
			$$
			\phi=\sum_{i=1}^3\phi(a_i)\phi_i+\sum_{i=1}^3\phi(m_i)\phi_{i+3}, 
			$$

			由积分公式
			$$
			\iint_{K}\lambda_{1}^{m}\lambda_{2}^{n}\lambda_{3}^{k}\mathrm{d}x\mathrm{d}y=2S_K\frac{m!\cdot n!\cdot k!}{(m+n+k+2)!}.
			$$
			
			代入计算可得
			
				$$
			\int_K \phi \mathrm{dx} \approx \frac{S_K}{3} \sum_{i=4}^6 \phi\left(a_i\right)
			$$
		\end{proof}
		\item 写出Linbo Zhang 2009论文中六点积分公式, 并说明阶数(积分点和权重可保留8位小数)
		
		$$
		|T|\sum_{i=1}^6f(p_i)w_i=\int_Tf(x)dx
		$$
		
		其中,积分点、权重以及对应阶数由下列表格给出(积分点由重心坐标表示,轮换对称):
		
		\begin{table}[h]
			\centering
			\begin{tabular}{|ccc|}
				\hline
				\multicolumn{3}{|c|}{\textbf{6-point order 3 rule on triangle}}                                                                 \\ \hline
				\multicolumn{1}{|c|}{Orbit} & \multicolumn{1}{c|}{Abscissas}                                                       & Weight     \\ \hline
				\multicolumn{1}{|c|}{$S_{111}$}  & \multicolumn{1}{c|}{\begin{tabular}[c]{@{}c@{}}0.23193337\\ 0.10903901\end{tabular}} & 0.16666667 \\ \hline
			\end{tabular}
		\end{table}
		
		\begin{table}[h]
			\centering
			\begin{tabular}{|ccc|}
				\hline
				\multicolumn{3}{|c|}{\textbf{6-point order 4 rule on triangle}}            \\ \hline
				\multicolumn{1}{|c|}{Orbit} & \multicolumn{1}{c|}{Abscissas}  & Weight     \\ \hline
				\multicolumn{1}{|c|}{$S_{21}$}   & \multicolumn{1}{c|}{0.09157621} & 0.10995174 \\ \hline
				\multicolumn{1}{|c|}{$S_{21}$}   & \multicolumn{1}{c|}{0.44594849} & 0.22338159 \\ \hline
			\end{tabular}
		\end{table}
	\end{enumerate}
	
	
\end{document}

