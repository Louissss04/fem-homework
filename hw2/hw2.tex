\documentclass[12pt,a4paper]{article}
\usepackage{geometry}
\geometry{left=2.5cm,right=2.5cm,top=2.0cm,bottom=2.5cm}
\usepackage[english]{babel}
\usepackage{amsmath,amsthm}
\usepackage{amsfonts}
\usepackage[longend,ruled,linesnumbered]{algorithm2e}
\usepackage{fancyhdr}
\usepackage{ctex}
\usepackage{array}
\usepackage{listings}
\usepackage{color}
\usepackage{graphicx}
\usepackage{amssymb}
\newtheorem{theorem}{定理}
\newtheorem{lemma}[theorem]{引理}
\newtheorem{corollary}[theorem]{推论}

\begin{document}
	
	\noindent
	
	\section*{2024.04.15}	
	
	
	\begin{enumerate}
		\item 推导		
		$
		\displaystyle \int_\Omega v\Delta^2u\mathrm{~dx}-\displaystyle \int_\Omega\Delta u\Delta v\mathrm{~dx}
		$
		边界项
		
		\begin{proof}[解]\let\qed\relax
			给定 $u,v\in H^1(\Omega)$, 根据Green 公式:对 $1\leqslant i\leqslant n$,
			\begin{equation}
				\int_{\Omega}u\partial_{i}v\mathrm{d}\boldsymbol{x}=-\int_{\Omega}v\partial_{i}u\mathrm{d}\boldsymbol{x}+\int_{\Gamma}uvn_{i}\mathrm{d}s\label{green1}
			\end{equation}
			
			若$u\in H^2(\Omega),\quad v\in H^1(\Omega)$,则可用 $\partial_{i}u$ 代替\eqref{green1}式中的 $u$,可得
			
			$$
			\int_{\Omega}\partial_{i}u\partial_{i}v\mathrm{d}\boldsymbol{x}=-\int_{\Omega}v\partial_{ii}u\mathrm{d}\boldsymbol{x}+\int_{\Gamma}\partial_{i}uvn_{i}\mathrm{d}s
			$$
			
			对 $i$ 从 1 到 $n$ 求和可得
			\begin{equation}
				\int_{\Omega}\nabla u\nabla v\mathrm{d}\boldsymbol{x}=-\int_{\Omega}v\Delta u\mathrm{d}\boldsymbol{x}+\int_{\Gamma}v\frac{\partial u}{\partial n}\mathrm{d}s\label{green2}
			\end{equation}
			
			若$v\in H^2(\Omega),\quad u\in H^1(\Omega)$,类似地有
			\begin{equation}
				\int_{\Omega}\nabla u\nabla v\mathrm{d}\boldsymbol{x}=-\int_{\Omega}u\Delta v\mathrm{d}\boldsymbol{x}+\int_{\Gamma}u\frac{\partial v}{\partial n}\mathrm{d}s\label{green3}
			\end{equation}
			
			则当$u, v\in H^2(\Omega)$,\eqref{green2}\eqref{green3}两式相减,得:
			\begin{equation}
				\int_{\Omega}\big(u\Delta v-v\Delta u\big)\mathrm{d}\boldsymbol{x}=\int_{\Gamma}\big(u\frac{\partial v}{\partial n}-v\frac{\partial u}{\partial n}\big)\mathrm{d}s,\forall u,v\in H^{2}(\Omega)\label{green4}
			\end{equation}
			
			若$u\in H^{4}(\Omega),v\in H^{2}(\Omega).$, 则可用 $\Delta u$ 代替\eqref{green4}中的 $u$, 可得
			
			$$
			\int_\Omega\Delta u\Delta v\mathrm{d}\boldsymbol{x}=\int_\Omega v\Delta^2u\mathrm{d}\boldsymbol{x}-\int_\Gamma v\frac{\partial\Delta u}{\partial n}\mathrm{d}s+\int_\Gamma\Delta u\frac{\partial v}{\partial n}\mathrm{d}s
			$$
			
		\end{proof}
		
		\item 
		当双线性型$a(\cdot,\cdot)$对称时,利用$a(\cdot,\cdot)$可定义$V$上的新内积和Riesz表示定理证明Lax-Milgram定理(根据注的提示)
		
		\begin{theorem}(Lax-Milgram 定理)
			
			设$V$ 是一个 Hilbert 空间,$a(u,v)$ 是$V\times V$ 上的对称、连续、强制的双线性形式,$f$ 是$V$ 中的线性连续泛函,则变分问题
			$$\text{求 }u\in V,\text{ 使得 }a(u,v)=\langle f,v\rangle,\quad\forall v\in V,$$
			
			存在唯一解$u^*$,且满足范数估计
			
			$$
			\left\|u^*\right\|_V\leqslant\frac{1}{\alpha}\left\|f\right\|_{V^*},
			$$

		\end{theorem}
		
		\begin{proof}[证明]
			因 $a(u,v)$ 是对称、正定的,故可在$V$ 上定义新内积 $[u,v]\triangleq a(u,v)$, 且
			
			$$
			\alpha{\left\|u\right\|}_{V}^{2}\leqslant\left[u,u\right]\leqslant M{\left\|u\right\|}_{V}^{2},
			$$
			
			其中左侧由强制性,右侧由连续性.
			
			新定义的内积所确定的范数 $\sqrt{a(u,u)}$ 与范数 $\|\cdot\|_V$ 等价. 对于 $V$ 的新范数$\sqrt{a(u,u)}$ 而言 $f$ 仍是线性连续泛函. 根据 Riesz 表示定理可知,存在唯一的 $u^*\in V$ 使得
			
			$$
			[u^*,v]=\left\langle f,v\right\rangle,\quad\forall\:v\in V,
			$$
			
			即
			
			$$
			a(u^*,v)=\langle f,v\rangle,\quad\forall\:v\in V.
			$$
			
			于是 $u^*$ 就是变分问题的唯一解.另一方面
			
			$$
			\alpha\|u^*\|_V^2\leqslant a(u^*,u^*)=[u^*,u^*]=\langle f,u^*\rangle\leqslant\|f\|_{V^*}\|u^*\|_V.
			$$
			
			从而得到范数估计.
		\end{proof}
		
		\item 证明Poincaré不等式
		\begin{equation}
			\|v\|_{m,\Omega}^2\leq k\Big(|v|_{m,\Omega}^2+\sum_{|\alpha|<m}\Big(\int_\Omega\partial^\alpha v\mathrm{dx}\Big)^2\Big),\forall v\in H^m(\Omega),\label{poincare}
		\end{equation}
		
		\begin{proof}[证明]
			利用反证法. 假设\eqref{poincare}不成立. 对每个正整数$k$, 存在$v_k\in H^m(\Omega)$使
			
			$$
			\|v_k\|^2_{m,\Omega}>C_2\Big(|v_k|_{m,\Omega}^2+\sum_{|\alpha|<m}\Big(\int_\Omega\partial^\alpha v_k\mathrm{dx}\Big)^2\Big).
			$$
			
			不失一般性,设$v_k$满足
			
			$$
			\left\|v_k\right\|_{m,\Omega}=1,\:k=1,2,\cdots,
			$$
			
			这样有
			
			$$
			\lim_{k\to\infty}|v_k|_{m,\Omega}=0.
			$$
			$$
			\lim_{k\to\infty}\int_\Omega\partial^\alpha v_k\mathrm{dx} = 0, \quad \forall \alpha \quad s.t. |\alpha|<m
			$$
			
			因为$\{v_k\}$是$H^m(\Omega)$的有界序列,所以存在$v_\infty\in H^m(\Omega)$和一个子列(仍记为$\{v_k\})$满足:$\{v_k\}$弱收敛于$v_\infty$. 利用嵌入定理$H^m(\Omega)\overset{c}{\operatorname*{\hookrightarrow}}H^{m-1}(\Omega)$,
			在$H^{m-1}(\Omega)$中$\{v_k\}$强收敛于$v_{\infty}.$
			
			于是
			
			$$
			\lim_{k,\ell\to\infty}\|v_k-v_\ell\|_{m,\Omega}\leq\lim_{k,\ell\to\infty}\left(\|v_k-v_\ell\|_{m-1,\Omega}+|v_k-v_\ell|_{m,\Omega}\right)=0.
			$$
			因此$\{v_k\}$是$H^m(\Omega)$中的Cauchy序列,进而$\{v_k\}$在$H^m(\Omega)$中强收敛于$v_\infty.$ 由极限得到$\|v_\infty\|_{m,\Omega}=1$和$|v_\infty|_{m,\Omega}=0$, 即$v_{\infty}$是一个$m-1$次多项式. 
			
			注意到
			$$
			\int_\Omega\partial^\alpha v_\infty \mathrm{dx} = 0, \quad \forall \alpha \quad s.t. \ |\alpha|<m
			$$,
			
			可得$v_\infty\equiv0$.与$v_\infty\neq0$矛盾.上面讨论可推出Poincaré不等式成立.
			
			
		\end{proof}
		
		\item 考虑Robin边值问题
		\begin{equation}
			\begin{cases}-\Delta u+\alpha(x)u=f,\text{ 在}\Omega \text{上}\\(\frac{\partial u}{\partial n}+\beta(x)u)\left.\right|_{\partial\Omega}=g&\end{cases}
			\label{robin}
		\end{equation}
		
		
		这里$\alpha(x)\geq\alpha_0>0,\beta(x)\geq0$以及$f$均为足够光滑的已知函数
		
		\begin{enumerate}
			\item[(i)]   试建立该边值问题的变分问题
			\item[(ii)]  讨论变分问题解的存在唯一性(适定性)
		\end{enumerate}
		\begin{proof}[解]\let\qed\relax
			\begin{enumerate}
				\item[(i)] 
				
				对任意 $v \in H^1(\Omega)$, 用Green公式可得
				$$
				\begin{aligned}
					& \int_{\Omega} \nabla u \cdot \nabla v + \alpha(x)uv \mathrm{dx}-\int_{\partial \Omega} \frac{\partial u}{\partial n} v \mathrm{ds}=\int_{\Omega} f v \mathrm{dx} \\
					& \Rightarrow  \int_{\Omega} \nabla u \cdot \nabla v + \alpha(x)uv \mathrm{dx}+\int_{\partial \Omega}\beta(x)uv \mathrm{ds}=\int_{\Omega} f v \mathrm{dx} + \int_{\partial \Omega} gv \mathrm{ds}
				\end{aligned}
				$$
				
				则问题\eqref{robin}的变分形式是: 求 $u \in H^1(\Omega)$ 使得
				\begin{equation}
					a(u, v)=(f, v)+\int_{\partial \Omega} gv \mathrm{ds}, \quad \forall v \in V.
					\label{weak}
				\end{equation}
				
				其中$a(u, v):= \int_{\Omega} \nabla u \cdot \nabla v + \alpha(x)uv \mathrm{dx}+\int_{\partial \Omega}\beta(x)uv \mathrm{ds}$.
				
				\item[(ii)] 只需要验证Lax-Milgram定理的条件:
				\begin{equation*}
					\begin{aligned}
						a(u,u) &= \int_{\Omega} (\nabla u)^2 + \alpha(x)u^2 \mathrm{dx}+\int_{\partial \Omega}\beta(x)u^2 \mathrm{ds}\\
						&\geqslant \int_{\Omega} (\nabla u)^2 + \alpha(x)u^2 \mathrm{dx}\\
						&\geqslant \min\{1,\alpha_0\} \|u\|_{1,\Omega}^2
					\end{aligned}
				\end{equation*}
				从而$a(u,v)$满足强制性
				
				\begin{equation*}
					\begin{aligned}
						|a(u,v)|&\leqslant\left|u\right|_{1,\Omega}\cdot\left|v\right|_{1,\Omega}+\sup_{\boldsymbol{x}\in\Omega}\alpha(\boldsymbol{x})\cdot\left\|u\right\|_{0,\Omega}\cdot\left\|v\right\|_{0,\Omega}+\sup_{\boldsymbol{x}\in\Omega}\beta(\boldsymbol{x})\cdot\left\|u\right\|_{0,\partial \Omega}\cdot\left\|v\right\|_{0,\partial \Omega}\\
						&\leqslant (1+\sup_{\boldsymbol{x}\in\Omega} \alpha(\boldsymbol{x})+\sup_{\boldsymbol{x}\in\Omega} \beta(\boldsymbol{x}))\|u\|_{1,\Omega}\cdot\|v\|_{1,\Omega},\quad\forall u,v\in V
					\end{aligned}
				\end{equation*}
				其中上式第一行用到了柯西积分不等式,第二行用到了范数性质以及迹定理.从而$a(u,v)$满足连续性.
			$$
			\left|\int_\Omega fv\mathrm{dx}\right|\leq\left\|f\right\|_{0,\Omega}\left\|v\right\|_{0,\Omega}\leq\left\|f\right\|_{0,\Omega}\left\|v\right\|_{1,\Omega}
			$$
			$$
			\left|\int_{\Omega}gv\mathrm{ds}\right|\leq\left\|g\right\|_{0,\partial\Omega}\left\|v\right\|_{0,\partial\Omega}\leq C\left\|g\right\|_{0,\partial\Omega}\left\|v\right\|_{1,\Omega}
			$$
			从而得到右端泛函有界性.
			
			综上,利用Lax-Milgram定理,变分问题\eqref{weak}的解存在且唯一.
			\end{enumerate}
		\end{proof}
		
	\end{enumerate}
	
	
	
\end{document}

