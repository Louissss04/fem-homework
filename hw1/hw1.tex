\documentclass[12pt,a4paper]{article}
\usepackage{geometry}
\geometry{left=2.5cm,right=2.5cm,top=2.0cm,bottom=2.5cm}
\usepackage[english]{babel}
\usepackage{amsmath,amsthm}
\usepackage{amsfonts}
\usepackage[longend,ruled,linesnumbered]{algorithm2e}
\usepackage{fancyhdr}
\usepackage{ctex}
\usepackage{array}
\usepackage{listings}
\usepackage{color}
\usepackage{graphicx}
\usepackage{amssymb}
\newtheorem{theorem}{定理}
\newtheorem{lemma}[theorem]{引理}
\newtheorem{corollary}[theorem]{推论}

\begin{document}
	
	\noindent
	
	\section*{2024.04.08}	
	

	\begin{enumerate}
		\item 利用$Young$不等式证明$H\ddot{o}lder$不等式.
		
		\begin{theorem}{($Young$不等式)}
			
			设 $p>1, q>1$, 且 $\frac{1}{p}+\frac{1}{q}=1$, 则对 $\forall a, b \geq 0$, 有
			\begin{equation} 
				a b \leq \frac{a^p}{p}+\frac{b^q}{q} .\label{young}
			\end{equation}
		\end{theorem}
		
		\begin{theorem}{($H\ddot{o}lder$不等式)}
		
		对 $f \in L^p(\Omega), g \in L^q(\Omega), 1 \leqslant p, q \leqslant+\infty, \frac{1}{p}+\frac{1}{q}=1$, 有
		\begin{equation}
			\|f g\|_{0,1, \Omega} \leqslant\|f\|_{0, p, \Omega}\|g\|_{0, q, \Omega} .\label{holder}
		\end{equation}
		
		
		\end{theorem}
		
		\begin{proof}[证明]
		
		若 $\|f\|_{0, p, \Omega}=0$ (或 $\|g\|_{0, q, \Omega}=0$), 则$f(\boldsymbol{x}) = 0$(或 $g(\boldsymbol{x})=0$), a.e. $\boldsymbol{x} \in \Omega$, 进而 $f(\boldsymbol{x}) g(\boldsymbol{x})=0$, a.e. $\boldsymbol{x} \in \Omega$, 此时 \eqref{holder}左侧与右侧均为0, 成立.
			
		否则 $\|f\|_{0, p, \Omega} \neq 0,\|g\|_{0, q, \Omega} \neq 0$. 取 $a=\frac{|f(\boldsymbol{x})|}{\|f\|_{0, p, \Omega}}, b=\frac{|g(\boldsymbol{x})|}{\|g\|_{0, q, \Omega}}$, 代入 \eqref{young}式后两边在 $\Omega$ 上积分, 得到 
		
		\begin{equation*}
			\begin{aligned}
				\int_{\Omega}\frac{|f(\boldsymbol{x})|}{\|f\|_{0, p, \Omega}}\frac{|g(\boldsymbol{x})|}{\|g\|_{0, q, \Omega}} dx &\leq \frac{1}{p} \frac{\int_{\Omega}|f(\boldsymbol{x})|^p dx}{\|f\|_{0, p, \Omega}^p} + \frac{1}{q} \frac{\int_{\Omega}|g(\boldsymbol{x})|^q dx}{\|g\|_{0, q, \Omega}^q}\\
				& = \frac{1}{p}+\frac{1}{q}\\
				& = 1
			\end{aligned} 
		\end{equation*}
		整理即得\eqref{holder}.
		
		\end{proof}
		
		\item 证明$W^{1,p}$是Banach空间.(利用$L^p$完备)
		
		\begin{proof}[证明]
			
			直接证明$W^{m,p}$是Banach空间.只要证明 $W^{m, p}(\Omega)$ 在Sobolev范数下是完备的.
			
			令 $\left\{v_j\right\} \subset W^{m, p}(\Omega)$ 是 Cauchy 列, 即 $\left\{D^\alpha v_j:|\alpha| \leq m\right\}$ 是 $L^p(\Omega)$ 中的 Cauchy 列. 由于 $L^p(\Omega)$ 是完备的, 从而存在 $v_\alpha \in L^p(\Omega)(|\alpha| \leq$ $m)$, 使得 $D^\alpha v_j \rightarrow v_\alpha$ 在 $L^p(\Omega)$ 中, 当 $j \rightarrow \infty$ 时. 余下只要证明 $v_\alpha=D^\alpha v$, 即 $\forall \varphi \in C_0^{\infty}(\Omega)$ ,
			$$
			\int_{\Omega} v_\alpha \cdot \varphi d x=(-1)^{|\alpha|} \int_{\Omega} v \cdot \partial^\alpha \varphi d x .
			$$
			
			事实上,
			$$
			\begin{aligned}
				\int_{\Omega} v_\alpha \cdot \varphi d x & =\lim _{j \rightarrow \infty} \int_{\Omega} D^\alpha v_j \cdot \varphi \mathrm{d} x \\
				& =\lim _{j \rightarrow \infty}(-1)^{|\alpha|} \int_{\Omega} v_j \cdot \partial^\alpha \varphi d x=(-1)^{|\alpha|} \int_{\Omega} v \cdot \partial^\alpha \varphi d x.
			\end{aligned}
			$$
		\end{proof}
		
		\item 设 $\Omega=(-1,1)$ ,
		$$
		f(x)=|x|= \begin{cases}x, & 0 \leq x<1 \\ -x, & -1<x<0\end{cases}
		$$
		
		证明
		$$
		g(x)= \begin{cases}1, & 0<x<1 \\ -1, & -1<x<0\end{cases}
		$$
		
		是 $f$ 的一阶广义导数.
		
		\begin{proof}[证明]
			$\forall \varphi \in C_0^{\infty}(\Omega)$, 由分部积分,
			$$
			\begin{aligned}
				\int_{-1}^1 f(x) \cdot \varphi^{\prime}(x) \mathrm{d} x & =\int_{-1}^0 f(x) \cdot \varphi^{\prime}(x) \mathrm{d} x+\int_0^1 f(x) \cdot \varphi^{\prime}(x) \mathrm{d} x \\
				& =\int_{-1}^0 \varphi(x) \mathrm{d} x-\int_0^1 \varphi(x) \mathrm{d} x \\
				& =-\int_{-1}^1 g(x) \varphi(x) \mathrm{d} x
			\end{aligned}
			$$
			
			其中
			$$
			g(x)=|x|= \begin{cases}1, & 0<x<1 \\ -1, & -1<x<0\end{cases}
			$$
			
			显然$g$是局部Lesbegue可积函数,故$g$是$f$的一阶广义导数.
		\end{proof}
		
		\item 推导Poisson方程的混合边值问题
		\begin{equation} 
			\left\{\begin{array}{l}
				-\Delta u=f, \text { 在 } \Omega \text { 内 } \\
				\left.u\right|_{\Gamma_1}=g_1 \\
				\left.\frac{\partial u}{\partial n}\right|_{\Gamma_2}=g_2
			\end{array}\right. \label{Possion}
		\end{equation}
		
		
		
		的变分形式, 这里 $\partial \Omega=\Gamma_1 \cup \Gamma_2$, 且 $\Gamma_1 \cap \Gamma_2=\emptyset$. 并证明古典解和弱解在一定条件下等价.
		
		\begin{proof}[解]\let\qed\relax
			设$V = \{v\ | v \in H^1(\Omega), \left.v\right|_{\Gamma_1} = 0\}$.
			
			对任意 $v \in V$, 用Green公式可得
			$$
			\begin{aligned}
				& \int_{\Omega} \nabla u \cdot \nabla v \mathrm{dx}-\int_{\Gamma_2} \frac{\partial u}{\partial n} v \mathrm{ds}=\int_{\Omega} f v \mathrm{dx} \\
				& \Rightarrow \int_{\Omega} \nabla u \cdot \nabla v \mathrm{dx}=\int_{\Omega} f v \mathrm{dx}+\int_{\Gamma_2} g_2 v \mathrm{ds}
			\end{aligned}
			$$
			
			则问题\eqref{Possion}的变分形式是: 求 $u \in H^1(\Omega)$且$\left.u\right|_{\Gamma_1} = g_1$ 使得
			\begin{equation}
				a(u, v)=(f, v)+\int_{\Gamma_2} g_2 v \mathrm{ds}, \quad \forall v \in V.
				\label{weak}
			\end{equation}
			
	
		
	\end{proof}
	
	\begin{theorem}
		设 $f \in C(\overline{\Omega}), g_1,g_2  \in C(\partial \Omega)$. 如果 $u \in C^2(\overline{\Omega})$ 是问题\eqref{Possion}的古典解, 则它是弱解. 反过来, 如果 $u$ 是问题\eqref{Possion}的弱解且 $u \in C^2(\overline{\Omega})$, 则它是古典解.
	\end{theorem}
	
	\begin{proof}[证明]
		第一个结论显然. 
		
		如果 $u$ 是问题\eqref{Possion}的弱解且 $u \in C^2(\overline{\Omega})$,立得$\left.u\right|_{\Gamma_1}=g_1$.
		
		第一步在\eqref{weak}中取 $v \in C_0^{\infty}(\Omega)$, 利用Green公式和 $\left.v\right|_{\partial \Omega}=0$ 得到
		$$
		\int_{\Omega}(\Delta u+f) v \mathrm{dx}=0 .
		$$
		
		推出 $\Delta u+f=0$. 第二步在(4)中取 $v \in C^{\infty}(\overline{\Omega})$且$\left.v\right|_{\Gamma_1} = 0$, 注意到 $\Delta u+f=0$, 可得
		$$
		\int_{\Gamma_2} \frac{\partial u}{\partial n} v \mathrm{ds}=\int_{\Gamma_2} g_2 v \mathrm{ds}
		$$
		
		由 $v$ 的任意性得到 $u$ 满足边界条件 $\left.\frac{\partial u}{\partial n}\right|_{\Gamma_2}=g_2$.
		
		三个条件均满足,故 $u$ 是古典解.
	\end{proof}
	\end{enumerate}
	
		
	
\end{document}

